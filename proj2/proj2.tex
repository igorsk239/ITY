\documentclass[a4paper, titlepage, twocolumn, 11pt]{article}
\usepackage[left=1.5cm,text={18cm, 25cm}, top=2.5cm]{geometry}
\usepackage{times}
\usepackage[IL2]{fontenc}
\usepackage[czech]{babel}
\usepackage[utf8]{inputenc}
\usepackage{amsthm, amssymb, amsmath}
\usepackage{stmaryrd}
\usepackage{nameref}
\usepackage{setspace}

\setlength{\skip\footins}{2pt}

\theoremstyle{definition}
\newtheorem{defn}{Definice}

\theoremstyle{sentence}
\newtheorem{sent}{Věta}


\begin{document}
  \begin{titlepage}
    \begin{center}
        \Huge \textsc{Fakulta informačních technologií\\ Vysoké učení technické v Brně}\\
        \LARGE
        \vspace{\stretch{0.382}}
        Typografie a publikování\,--\,2. projekt\\
        Sazba dokumentů a matematických výrazů
        \vspace{\stretch{0.618}}
    \end{center}
    {\Large 2019 \hfill Igor Ignác (xignac00)}
  \end{titlepage}

\section*{Úvod}
\setlength{\parindent}{1em}
\label{fig:pageone}
V~této úloze si vyzkoušíme sazbu titulní strany, matematických vzorců, prostředí a~dalších textových struktur obvyklých pro technicky zaměřené texty (například rovnice~(\ref{arr:first}) nebo Definice \ref{def:defone} na straně~\pageref{fig:pageone}). Pro odkazovaní na vzorce a~struktury zásadně používáme příkaz \verb|\label| a~\verb|\ref| případně\verb|\pageref| pokud se chceme odkázat na stranu výskytu.
\par
Na titulní straně je využito sázení nadpisu podle optického středu s~využitím zlatého řezu. Tento postup byl probírán na přednášce. Dále je použito odřádkování sezadanou relativní velikostí 0.4 em a 0.3 em.

\section{Matematický text}
Nejprve se podíváme na sázení matematických symbolů a~výrazů v~plynulém textu včetně sazby definic a~vět s~využitím balíku \verb|amsthm|. Rovněž použijeme poznámku pod čarou s~použitím příkazu \verb|\footnote|. Někdy je vhodné použít konstrukci \verb|\mbox{}|, která říká, že text nemá být zalomen.
\par
\begin{defn}\label{def:defone}
Zásobníkový automat (ZA) \emph{je definován jako sedmice tvaru} $A$ $=$ ($Q$, $\Sigma$, $\Gamma$, $\delta$, $q_0$, $Z_0$, $F$)\emph{, kde: }

  \begin{itemize}
    %\setlength\itemsep{0em}
    \item $Q$ \emph{je konečná množina} vnitřních (řídicích) stavů,
    \item $\Sigma$ \emph{je konečná} vstupní abeceda,
    \item $\Gamma$ \emph{je konečná} zásobníková abeseda,
    \item $\delta$ \emph{je} přechodová funkce $Q\times(\Sigma \cup {\{\epsilon\}}) \times \Gamma\shortrightarrow 2^{Q\times\Gamma^\ast}$
    \item $q_0 \in Q$ \emph{je} počáteční stav, $Z_0$ $\in$ $\Gamma$ \emph{je} startovací symbol zásobníku \emph{a $F$ $\subseteq$ $Q$ množina} koncových stavů.

  \end{itemize}

  Nechť $P = (Q, \Sigma, \Gamma, \delta, q_0, Z_0, F)$ je zásobníkový automat. \emph{Konfigurací} nazveme trojici $(q, w, \alpha) \in Q \times \Sigma^\ast \times \Gamma^\ast$, kde $q$ je aktuální stav vnitřního řízení, $w$ je dosud nezpracovaná část vstupního řetězce a~$\alpha = Z_{i_1} Z_{i_2}$ \ldots $Z_{i_k}$ je obsah zásobníku\footnotemark[1].

\end{defn}
\subsection{Podsekce obsahující větu a odkaz}
\begin{defn}\label{def:deftwo}
Řetězec $w$~nad abecedou $\Sigma$ je přijat ZA $A$~\emph{jestliže} $(q_0, w, Z_0) \overset{\ast}{\underset{A}{\vdash}} (q_F, \epsilon, \gamma)$ \emph{pro nějaké} $\gamma \in \Gamma^\ast$ \emph{a} $q_F \in F$. \emph{Množinu} $L(A) = \{$ $w$ $|$ $w $ \emph{je přijat ZA A} $\}$ $\subseteq \Sigma^\ast$ \emph{nazývamé} jazyk přijímaný TS $M$.
\par
\setlength{\parskip}{1em}
Nyní si vyzkoušíme sazbu vět a~důkazů opět s~použitím balíku \verb|amsthm|.
\end{defn}
\footnotetext[1]{$Z_{i_1}$ je vrchol zásobníku}
\medbreak

\begin{sent}
  Třída jazyků, které jsou přijímany ZA, odpovídá \emph{bezkontextovým jazykům}.
\end{sent}
\setlength\parindent{0pt}
\emph{Důkaz}. V důkaze vyjdeme z Definice \ref{def:defone} a \ref{def:deftwo}.\hfill\ensuremath{\square}
\section{Rovnice a odkazy}
Složitější matematické formulace sázíme mimo plynulý text. Lze umístit několik výrazů na jeden řádek, ale pak je třeba tyto vhodně oddělit, například příkazem \verb|\quad|.\\\\
$\sqrt[i]{x_i^3}$\quad  kde $x_i$ je $i$-té sudé číslo splňující\quad $x_i^{2-x_i^{i^2}} \leq x_i^{y_i^3}$
\setlength{\parindent}{1em}
\setlength{\parskip}{0.4em}
\par
V~rovnici (\ref{arr:first}) jsou využity tři typy závorek s~různou explicitně definovanou velikostí.

\begin{eqnarray}\label{arr:first}
  x&=&\bigg[\Big\{\big[a + b\big] * c\Big\}^d \ominus 1\bigg]^{1/2}\\
  y&=&\lim_{x \rightarrow \infty} \frac{\frac{1}{\log_{10}x}}{\sin^2x + \cos^2x}\nonumber
\end{eqnarray}
\par
V~této větě vidíme, jak vypadá implicitní vysázení limity $\lim_{n \rightarrow \infty} f(n)$ v~normálním odstavci textu. Podobně je to i s~dalšími symboly jako $\prod_{i=1}^n 2^i$ či $\bigcap_{A\in\mathcal{B}}A$. V~případě vzorců $\displaystyle\lim_{n \rightarrow \infty} f(n)$ a~$\prod\limits_{i=1}^{n}2^i$ a~jsme si vynutili méně úspornou sazbu příkazem \verb|\limits|.

\begin{eqnarray}
  \int_{b}^{a}g(x)dx&=&-\int\limits_{a}^{b}f(x)\,dx\\
  \overline{\overline {A\wedge B}}&\Leftrightarrow&\overline{\overline A \vee \overline B}
\end{eqnarray}

\section{Matice}
Pro sázení matic se velmi často používá prostředí \verb|array| a~závorky (\verb|\left|, \verb|\right|).

$$\left[\begin{array}{ccc}  &\widehat{\beta+\gamma}&\hat{\mathbf{\pi}}\\
\vec{a}&\overleftrightarrow{AC}& \\
\end{array}
\right]=1\Longleftrightarrow\mathbb{Q}=\mathbf{R}$$

$$\textbf{A}=\left|\begin{array}{cccc} a_{11}&a_{12}&\ldots&a_{1n}\\
a_{21}&a_{22}&\ldots&a_{2n}\\
\vdots&\vdots&\ddots&\vdots\\
a_{m1}&a_{m2}&\ldots&a_{mn}\\
\end{array}\right|=\begin{array}{cc} t&u\\
v&w\\
\end{array}=tw-uv$$
Prostředí \verb|array| lze úspěšně využít i~jinde.
$$\binom{n}{k}=
\begin{cases}
\ 0& \text{pro } k < 0 \text{ nebo }k > n\\
\ \frac{n!}{k!(n-k)!} & \text{pro } 0 \leq k\:\leq n\\
\end{cases}$$
\end{document}
