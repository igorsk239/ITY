\documentclass[a4paper, titlepage, twocolumn, 11pt]{article}
\usepackage[left=1.5cm,text={18cm, 25cm}, top=2.5cm]{geometry}
\usepackage{times}
\usepackage[IL2]{fontenc}
\usepackage[czech]{babel}
\usepackage[utf8]{inputenc}
\usepackage{amsthm}
\usepackage{ stmaryrd }
%\usepackage{amsthm,amssymb,amsmath,enumerate}


\theoremstyle{definition}
\newtheorem{defn}{Definice}

% \usepackage[hidelinks]{hyperref}
% \usepackage{csquotes}

\begin{document}
  \begin{titlepage}
    \begin{center}
        \Huge \textsc{Fakulta informačních technologií\\ Vysoke učení technicke v Brně}\\
        \LARGE
        \vspace{\stretch{0.382}}
        Typografie a publikování\,--\,2. projekt\\
        Sazba dokumentů a matematických výrazů
        \vspace{\stretch{0.618}}
    \end{center}
    {\Large 2019 \hfill Igor Ignác (xignac00)}
  \end{titlepage}

\section*{Úvod}
\label{fig:pageone}
V~této úloze si vyzkoušíme sazbu titulní strany, matematických vzorců, prostředí a~dalších textových struktur obvyklých pro technicky zaměřené texty (například rovnice ?? nebo Definice~\ref{ch:defone} na straně~\pageref{fig:pageone}). Pro odkazovaní na vzorcea struktury zásadně používáme příkaz \verb|\label| a~\verb|\ref| případně\verb|\pageref| pokud se chceme odkázat na stranu výskytu.
\par
Na titulní straně je využito sázení nadpisu podle optického středu s~využitím zlatého řezu. Tento postup byl probírán na přednášce. Dále je použito odřádkování sezadanou relativní velikostí 0.4 em a 0.3 em.

\section{Matematický text}
Nejprve se podíváme na sázení matematických symbolů a~výrazů v~plynulém textu včetně sazby definic a~vět s~využitím balíku \verb|amsthm|. Rovněž použijeme poznámku podčarou s~použitím příkazu \verb|\footnote|. Někdy je vhodné použít konstrukci \verb|\mbox{}|, která říká, že text nemá být zalomen.
\par
\begin{defn}
Zásobníkový automat (ZA) \emph{je definován jako sedmice tvaru $A$ $=$ ($Q$, $\Sigma$, $\Gamma$, $\delta$, $q_0$, $Z_0$, $F$), kde: }

  \begin{itemize}
    %\setlength\itemsep{0em}
    \item $Q$ \emph{je konečná množina} vnitřních (řídicích) stavů,
    \item $\Sigma$ \emph{je konečná} vstupní abeceda,
    \item $\Gamma$ \emph{je konečná} zásobníková abeseda,
    \item $\delta$ \emph{je} přechodová funkce $Q$$\times$($\Sigma$$\cup${$\epsilon$})$\times$$\Gamma$$\shortrightarrow$, $2^{Q\times\Gamma^\ast}$
    \item $q_0$ $\in$ \emph{je} počáteční stav, $Z_0$ $\in$ $\Gamma$ \emph{je} startovací symbol zásobníku \emph{a $F$ $\subseteq$ $Q$ množina} koncových stavů.

    %\medbreak
  \end{itemize}
\end{defn}


\end{document}
